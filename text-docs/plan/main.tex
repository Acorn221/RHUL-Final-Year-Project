\documentclass{article}

% Language setting
\usepackage[english]{babel}

\usepackage[T1]{fontenc}
\usepackage{lmodern}
\usepackage{mathptmx}
\usepackage{titlesec}
\usepackage{lipsum}

% Setting the section title size and spacing
\titleformat{\section}{\LARGE\scshape}{\thesection}{1em}{}

% Set page size and margins
\usepackage[a4paper,top=2cm,bottom=2cm,left=3cm,right=3cm,marginparwidth=1.75cm]{geometry}

% Useful packages
\usepackage{amsmath}
\usepackage{graphicx}
\usepackage[colorlinks=true, allcolors=blue]{hyperref}

\font\titleFont=cmr12 at 26pt
\font\subtitleFont=cmr12 at 20pt

\title{\titleFont Comparison Of Machine Learning Algorithms: Comparing Pruning Algorithms on Models After Transfer Learning}
\date{\Large \today}
\author{\subtitleFont James Arnott}

\begin{document}

\maketitle

\begin {center}
\noindent\rule{15cm}{0.4pt}
\end{center}

\vspace{1cm}

\begin{center}
	\begin{Large}
			CS3821 - BSc Final Year Project

			\vspace{0.6cm}

			Supervised by Li Zhang

			\vspace{0.6cm}

			Department Of Computer Science

			\vspace{0.2cm}

			Royal Holloway, University Of London
	\end{Large}
\end{center}


\pagebreak

\section{Abstract}
My final year project will be comparing different pruning algorithms for neural nets,
after using transfer learning on a variety of different pre-trained image classification models.

Pruning is the process of removing nodes from a neural network.
This is done to reduce the size of the network, and therefore the amount of memory and
computation required to run it.

This is not just useful for lower powered embedded systems,
where processing power is very limited, but also for larger systems,
where massive neural networks can be used to achieve very
high accuracy in classification but can be optimised to run faster
to be deployed on mass in real world applications, to increase
commercial viability.

When using transfer learning with a pre-trained model, there can be many
unnessasary nodes in the network that are not required for the new task
which the transfer learning is being used for. This is where pruning can be used
to remove these nodes and reduce the size of the network, significantly optimising it.

There are many different pruning algorithms, and this project will hope to compare
them and their performance on a variety of different pre-trained models, after
transfer learning to identify a reduced set of entities.

I will need to consider very carefully what the entities the network will try
to identify are, as I will want to avoid any bias in the data set, and
ensure that the network is trained on a diverse set of images, whilst avoiding
overfitting. I will also need to ensure that there are similar entities that the
network can distinguish between, to ensure that the network is not too simple.



\pagebreak
\section{Timeline}

\lipsum[1]\lipsum[2]

\pagebreak
\section{Risks and Mitigations}

\lipsum[1]\lipsum[2]

\pagebreak
\section{References}

\lipsum[1]\lipsum[2]

\end{document}