\documentclass{article}

% Language setting
\usepackage[english]{babel}

\usepackage[T1]{fontenc}
\usepackage{lmodern}
\usepackage{mathptmx}
\usepackage{titlesec}
\usepackage{lipsum}
\usepackage{pgfgantt}

% Setting the section title size and spacing
\titleformat{\section}{\LARGE\scshape}{\thesection}{1em}{}

% Set page size and margins
\usepackage[a4paper,top=2cm,bottom=2cm,left=3cm,right=3cm,marginparwidth=1.75cm]{geometry}

% Useful packages
\usepackage{amsmath}
\usepackage{graphicx}
\usepackage[colorlinks=true, allcolors=blue]{hyperref}

\font\titleFont=cmr12 at 26pt
\font\subtitleFont=cmr12 at 20pt

\title{\titleFont Comparison Of Machine Learning Algorithms: Comparing Pruning Algorithms on Models After Transfer Learning}
\date{\Large \today}
\author{\subtitleFont James Arnott}

\begin{document}

\maketitle

\begin {center}
\noindent\rule{15cm}{0.4pt}
\end{center}

\vspace{1cm}

\begin{center}
	\begin{Large}
			CS3821 - BSc Final Year Project

			\vspace{0.6cm}

			Supervised by Li Zhang

			\vspace{0.6cm}

			Department Of Computer Science

			\vspace{0.2cm}

			Royal Holloway, University Of London
	\end{Large}
\end{center}


\pagebreak

\section{Abstract}
My final year project will be comparing different pruning algorithms for neural nets,
after using transfer learning on a variety of different pre-trained image classification models.

Pruning is the process of removing nodes from a neural network.
This is done to reduce the size of the network, and therefore the amount of memory and
computation required to run it.

This is not just useful for lower powered embedded systems,
where processing power is very limited, but also for larger systems,
where massive neural networks can be used to achieve very
high accuracy in classification but can be optimised to run faster
to be deployed on mass in real world applications, to increase
commercial viability.

When using transfer learning with a pre-trained model, there can be many
unnessasary nodes in the network that are not required for the new task
which the transfer learning is being used for. This is where pruning can be used
to remove these nodes and reduce the size of the network, significantly optimising it.

There are many different pruning algorithms, and this project will hope to compare
them and their performance on a variety of different pre-trained models, after
transfer learning to identify a reduced set of entities.

I will need to consider very carefully what the entities the network will try
to identify are, as I will want to avoid any bias in the data set, and
ensure that the network is trained on a diverse set of images, whilst avoiding
overfitting. I will also need to ensure that there are similar entities that the
network can distinguish between, to ensure that the network is not too simple.



\pagebreak
\section{Timeline}

\begin{ganttchart}{1}{27}
\gantttitle{Term 1}{27} \\
\gantttitlelist{1,2,3,4,5,6,7,8,9}{3} \\
\ganttgroup{\small Research}{1}{8} \\
\ganttgroup{\small Implementation}{9}{16} \\
\ganttbar{\small Classification Models}{1}{5} \\
\ganttlinkedbar{\small Pruning Algorithms}{3}{8} \ganttnewline 
\ganttlinkedbar{\small Classification models}{9}{11} \ganttnewline
\ganttlinkedbar{\small Transfer Learning}{10}{13} \ganttnewline
\ganttlinkedbar{\small Pruning Algorithms models}{12}{16} \ganttnewline
\ganttmilestone{\small Algorithms can be ran}{16} \ganttnewline
\ganttlinkedbar{\small Comparing Algorithms}{18}{25} \ganttnewline
\ganttlinkedbar{\small Report/Presentation}{25}{27}
\ganttlink{elem6}{elem7}
\end{ganttchart}

\subsection{Research Models}

Here, I will the pre-trained research models that I will use, along with the pruning algorithms I can implement with them. I will also research how to do transfer learning with the models I want to use for the comparisons.

\subsection{Implementation}

Here I will setup the models and make sure I can run them, the transfer learning will also be done here to specialise the models so they are trained to identify a specific subset of entities.

I will also setup the pruning algorithms to see how effective they are.

\subsection{Comparing Algorithms}

Here I will write scripts to compare the models after training and after pruning. I have allocated a lot of time to this as I will likely need to debug problems I have had with implementation, then I'll need to find good metrics which I can compare the models performance and efficiency with.

\subsection{Report/Presentation}

Here is where I will prepare the for the interim report and presentation.

\pagebreak
\begin{ganttchart}{1}{22}
\gantttitle{Term 2}{22} \\
\gantttitlelist{1,...,11}{2} \\
\ganttgroup{\small Evaluate The Comparison}{1}{8} \\
\ganttgroup{\small Optimise The Models}{9}{15} \\
\ganttbar{\small Compare Performance}{1}{5} \\
\ganttlinkedbar{\small Identify Problems}{3}{8} \ganttnewline 
\ganttlinkedbar{\small Try and Fix the problems}{9}{13} \ganttnewline
\ganttlinkedbar{\small Optimise For Efficiency}{10}{15} \ganttnewline
\ganttmilestone{\small MVP}{15} \ganttnewline
\ganttlinkedbar{\small Refactor the code}{16}{20} \ganttnewline
\ganttlinkedbar{\small Finish Report/Presentation}{18}{22}
\ganttlink{elem5}{elem6}
\end{ganttchart}

\subsection{Evaluate The Comparison}

Here I will try and find anomalies in my results and attempt to find fixes for the optimisation, I am likely to have misconfigured one of the models in the transfer learning or pruning and this is where I will hopefully identify where.

\subsection{Optimise The Models}

I will hopefully optimise the models here, after adjusting the transfer learning parameters and the pruning algorithm parameters. This may not be required however I assume with machine learning, there are always likely to be possible improvements.

\subsection{Finalising}

Refactoring the code should hopefully not be too complex and I should aim to keep everything in a relatively strict format throughout the project. The report I write and the presentation I make should be quite a big task with the time I have allocated, however it should be manageable and I will aim to keep my report updated throughout the project.

\pagebreak
\section{Risks and Mitigations}
With machine learning, I know there are many risks that many different aspects of training, pruning and testing can all go wrong.

\subsection{Problems with finding training data for transfer learning}
There can be problems with finding good data sets for the entities that I end up transfer learning with. As I have not decided what to use transfer learning for I can decide what would be the best data set and decide to transfer learn to that. This allows me to be more flexible and gives this project the highest chance of success.

\subsection{Problems with getting models to work}
As I want to use a few different models, the minimum ideally would be 10, I may have problems getting the models to work in the environment I want to use for all the models. This is not a huge issue as there are many tutorials online which I can follow to setup the models.

\subsection{Implementing the pruning algorithms}
This is the biggest risk to the project, I have very limited experience with machine learning and no experience at all with pruning models. I have a basic understanding about how pruning is supposed to work however I will be relying on tutorials primarily for the implementation of the algorithms and this is likely to be the most time consuming part of the project.

\subsection{Limitations of my available computing resources}
I may end up requiring extra computing resources for running some of the training and pruning algorithms 

\subsection{Problems with testing}
Throughout the project, I will need to compare the performance of the models, which is not too difficult as I can just have a test data set to compare them against. Testing the model's size would not be too difficult either as I could just use the model's file sizes, however, I am not yet confident as to how I will compare the models computational demand.



\pagebreak
\section{References}

\lipsum[1]\lipsum[2]

\end{document}